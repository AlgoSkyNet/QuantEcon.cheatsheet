\documentclass[]{article}
\usepackage{graphicx}
\usepackage{amssymb}
\usepackage{amsthm}
\usepackage{amsmath}
\usepackage{bm}
\usepackage{fullpage}
\usepackage{array}
\usepackage{longtable}
\usepackage{minted}
\usepackage[margin=2cm]{geometry}
\begin{document}

\title{%
\textbf{QuantEcon Julia Cheatsheet}}
\date{}
\maketitle

\section{Variables}

Here are a few examples of basic kinds of variables we might be interested in creating.
\begin{longtable}{ |m{6cm}  | m{11cm} |}
	\hline
	\textbf{Command} & \textbf{Description}
	\\\hline
\begin{minted}{julia}
A = 4.1
B = [1, 2, 3]
C = [1.1 2.2 3.3]
D = [1 2 3]'
E = [1 2; 3 4]
\end{minted}
	& How to \textbf{create a scalar, a vector, or a matrix}. Here, each example will result in a slightly different form of output. \texttt{A} is a scalar, \texttt{B} is a flat array with 3 elements, \texttt{C} is a 1 by 3 vector, \texttt{D} is a 3 by 1 vector, and \texttt{E} is a 2 by 2 matrix.
	\\\hline
\begin{minted}{julia}
s = "This is a string"
\end{minted}
	& A \textbf{string} variable 
	\\\hline
\begin{minted}{julia}
x = true
\end{minted}
	& A \textbf{Boolean} variable 
	\\\hline
\end{longtable}

\section{Vectors and Matrices}

These are a few kinds of special vectors/matrices we can create and some things we can do with them.
\begin{longtable}{ |m{6cm}  | m{11cm} |}
	\hline
	\textbf{Command} & \textbf{Description}
	\\\hline
	\begin{minted}{julia}
A = zeros(m, n)
\end{minted}
& Creates a \textbf{matrix of all zeros} of size \texttt{m} by \texttt{n}. We can also
do the following:
\begin{minted}{julia}
A = zeros(B)
\end{minted}
which will create a matrix of all zeros with the same dimensions as
matrix or vector \texttt{B}. 
\\\hline
\begin{minted}{julia}
A = ones(m, n)
\end{minted}
& Creates a \textbf{matrix of all ones} of size \texttt{m} by \texttt{n}. We can also
do the following:
\begin{minted}{julia}
A = ones(B)
\end{minted}
which will create a matrix of all ones with the same dimensions as
matrix or vector \texttt{B}. 
\\\hline
\begin{minted}{julia}
A = eye(n)
\end{minted}
& Creates an \texttt{n} by \texttt{n} \textbf{identity matrix}. For example,
\texttt{eye(3)} will return
$\begin{pmatrix}
1 & 0 & 0\\
0 & 1 & 0\\
0 & 0 & 1\\
\end{pmatrix}$.
\\\hline
\begin{minted}{julia}
A = j:k:n
\end{minted}
	& This will create a \textbf{sequence} starting at \texttt{j}, ending at
    \texttt{n}, with difference \texttt{k} between points. For example,
    \texttt{A = 2:4:10} will create the sequence \texttt{2, 6, 10}. To convert the 
    output to an array, use \texttt{collect(A)}.	
    \\\hline
\begin{minted}{julia}
A = linspace(j, n, m)
\end{minted}
	& This will create a \textbf{sequence} of \texttt{m} points starting at \texttt{j}, ending at
    \texttt{n}. For example,
    \texttt{A = linspace(2, 10, 3)} will create the sequence \texttt{2.0, 6.0,
    10.0}. To convert the 
    output to an array, use \texttt{collect(A)}.	
    \\\hline
\begin{minted}{julia}
A = diagm(x)
\end{minted}
& Creates a \textbf{diagonal matrix} using the elements in \texttt{x}.  For example, if
\texttt{x = [1, 2, 3]}, \texttt{diagm(x)} will return
$\begin{pmatrix}
1 & 0 & 0\\
0 & 2 & 0\\
0 & 0 & 3\\
\end{pmatrix}$.
\\\hline
\begin{minted}{julia}
A = rand(m, n)
\end{minted}
& Creates an \texttt{m} by \texttt{n} \textbf{matrix of random numbers} drawn from a
\textbf{uniform distribution} on $[0, 1]$. Alternatively, \texttt{rand} can be used to
draw random elements from a set \texttt{X}. For example, if \texttt{X = [1, 2,
3]}, \texttt{rand(X)} will return either \texttt{1}, \texttt{2}, or \texttt{3}.
\\\hline
\begin{minted}{julia}
A = randn(m, n)
\end{minted}
& Creates an \texttt{m} by \texttt{n} \textbf{matrix of random numbers} drawn form a
\textbf{standard normal distribution}.\\\hline
\begin{minted}{julia}
A[m, n]
\end{minted}
	& This is the general syntax for \textbf{accessing elements} of an array or matrix, where \texttt{m} and \texttt{n} are integers. The example here returns the element in the second row and third column. 
	\begin{itemize}
	\item We can also use ranges (like \texttt{1:3}) in place of single numbers to extract multiple rows or columns. 
	\item A colon, \texttt{:}, by itself indicates all rows or columns
	\item The word \texttt{end} can also be used to indicate the last row or column.
	\end{itemize}
	\\\hline
\begin{minted}{julia}
nrow, ncol = size(A)
\end{minted}
	& \textbf{Returns the number of rows and columns} in a matrix. Alternatively, we can do:
\begin{minted}{julia}
nrow = size(A, 1)
\end{minted}
	and 
\begin{minted}{julia}
ncol = size(A, 2)
\end{minted}
	\\\hline
\begin{minted}{julia}
diag(A)
\end{minted}
	& This function returns a vector of the \textbf{diagonal elements} of
    \texttt{A} (i.e., \texttt{A[1, 1], A[2, 2]}, etc...).
	\\\hline
\begin{minted}{julia}
A = hcat([1 2], [3 4])
\end{minted}
    & \textbf{Horizontally concatenates} two matrices or vectors. The example here
    would return 
    $\begin{pmatrix}
    1 & 2 & 3 & 4
    \end{pmatrix}$. An alternative syntax is:
    \begin{minted}{julia}
    A = [[1 2] [3 4]]
    \end{minted}
    For either of these commands to work, both matrices or vectors must have
    the same number of rows.
	\\\hline
\begin{minted}{julia}
A = vcat([1 2], [3 4])
\end{minted}
    & \textbf{Vertically concatenates} two matrices or vectors. The example here
    would return 
    $\begin{pmatrix}
    1 & 2 \\
    3 & 4
    \end{pmatrix}$. An alternative syntax is:
    \begin{minted}{julia}
    A = [[1 2]; [3 4]]
    \end{minted}
    For either of these commands to work, both matrices or vectors must have
    the same number of columns.
	\\\hline
\begin{minted}{julia}
A = reshape(a, m, n)
\end{minted}
    & \textbf{Reshapes} matrix or vector \texttt{a} into a new matrix or
    vector, \texttt{A} with \texttt{m} rows and \texttt{n} columns. For
    example, 
    \begin{minted}{julia}
    A = reshape(1:10, 5, 2)
    \end{minted}
    would return 
    $\begin{pmatrix}
    1 & 6 \\
    2 & 7 \\
    3 & 8 \\
    4 & 9 \\
    5 & 10
    \end{pmatrix}$
    For this to work, the number of elements in \texttt{a} (number of rows
    times number of columns) must equal \texttt{m * n}.
    \\\hline
\begin{minted}{julia}
A[:]
\end{minted}
    & \textbf{Converts matrix A to a vector.} For
    example, if \texttt{A = [1 2; 3 4]}, then \texttt{A[:]} will return
    $\begin{pmatrix}
        1 \\
        3 \\
        2 \\
        4
    \end{pmatrix}$.
    \\\hline
\begin{minted}{julia}
flipdim(A, d)
\end{minted}
    & Reverses the vector or matrix \texttt{A} along dimension \texttt{d}. For
    example, if \texttt{A = [1 2 3; 4 5 6]}, \texttt{flipdim(A, 1)}, will
    reverse the rows of \texttt{A} and return 
    $\begin{pmatrix}
    4 & 5 & 6 \\
    1 & 2 & 3
    \end{pmatrix}$.
    \texttt{flipdim(A, 2)} will reverse the columns of \texttt{A} and return
    $\begin{pmatrix}
    3 & 2 & 1 \\
    6 & 5 & 4
    \end{pmatrix}$.
    \\\hline
\begin{minted}{julia}
repmat(A, m, n)
\end{minted}
    & \textbf{Repeats matrix} \texttt{A}, \texttt{m} times in the row
    direction and \texttt{n} in the column direction. For example, if
    \texttt{A = [1 2; 3 4]}, \texttt{repmat(A, 2, 3)} will return
    $\begin{pmatrix}
    1 & 2 & 1 & 2 & 1 & 2 \\
    3 & 4 & 3 & 4 & 3 & 4 \\
    1 & 2 & 1 & 2 & 1 & 2 \\
    3 & 4 & 3 & 4 & 3 & 4 \\
    \end{pmatrix}$.
    \\\hline
\end{longtable}

\section{Mathematical Functions}
Here, we cover some useful functions for doing math.
\begin{longtable}{ |m{6cm}  | m{11cm} |}
	\hline
	\textbf{Command} & \textbf{Description}
	\\\hline
	\begin{minted}{julia}
5 + 2
5 - 2
5 * 2 
5 \ 2
5 ^ 2
5 % 2
	\end{minted}
	& \textbf{Scalar arithmetic operations}: addition, subtraction, multiplication, division, power, remainder.
	\\\hline
	\begin{minted}{julia}
A + B
A - B
A .* B
A ./ B
A .^ B
A .% B
	\end{minted}
    & \textbf{Element-by-element operations} on matrices. This syntax applies
    the operation element-wise to corresponding elements of the matrices.
    \\\hline
	\begin{minted}{julia}
A * B
	\end{minted}
	& When \texttt{A} and \texttt{B} are matrices, \texttt{*} will perform
    \textbf{matrix multiplication}, as long as the number of columns in \texttt{A} is the same as the number of columns in \texttt{B}. 
	\\\hline
	\begin{minted}{julia}
dot(A, B)
	\end{minted}
	& This function returns the \textbf{dot product/inner product} of the two vectors
    \texttt{A} and \texttt{B}. The two vectors need to be dimensionless or
    column vectors.	\\\hline
    \begin{minted}{julia}
A'
	\end{minted}
	& This syntax returns the \textbf{transpose} of the matrix
    \texttt{A} (i.e., reverses the dimensions of \texttt{A}). For example if
    $A = \begin{pmatrix}
    1 & 2 \\
    3 & 4 
    \end{pmatrix}$, then \texttt{A'} returns 
    $\begin{pmatrix}
    1 & 3 \\
    2 & 4
    \end{pmatrix}$. \\\hline
\begin{minted}{julia}
sum(A)
maximum(A)
minumum(A)
	\end{minted}
	& These functions compute the sum, maximum, and minimum elements,
    respectively, in matrix or vector \texttt{A}. We can also add an
    additional argument for the dimension to compute the sum/maximum/minumum
    across. For example \texttt{sum(A, 2)} will compute the row sums of
    \texttt{A} and \texttt{maximum(A, 1)} will compute the maxima of each
    column of \texttt{A}.
    \\\hline
    \begin{minted}{julia}
inv(A)
	\end{minted}
	& This function returns the \textbf{inverse} of the matrix
    \texttt{A}. Alternatively, we can do:
    \begin{minted}{julia}
    A ^ (-1)
    \end{minted}
    \\\hline
 \begin{minted}{julia}
det(A)
	\end{minted}
	& This function returns the \textbf{determinant} of the matrix
    \texttt{A}.     
    \\\hline
 \begin{minted}{julia}
val, vec = eig(A)
	\end{minted} 
&    Returns the \textbf{eigenvalues} (\texttt{val}) and \textbf{eigenvectors} (\texttt{vec}) of
    matrix \texttt{A}. In the output, \texttt{val[i]} is the eigenvalue
    corresponding to eigenvector \texttt{val[:, i]}.
    \\\hline
 \begin{minted}{julia}
norm(A)
	\end{minted} 
&    Returns the Euclidean \textbf{norm} of matrix or vector \texttt{A}.
We can also provide an argument \texttt{p}, like so:
\begin{minted}{julia}
norm(A, p)
\end{minted}
which will compute the \texttt{p}-norm (the default \texttt{p} is 2). If
\texttt{A} is a matrix, valid values of \texttt{p} are \texttt{1, 2} amd
\texttt{Inf}.
\\\hline
 \begin{minted}{julia}
A \ b
	\end{minted} 
&   If \texttt{A} is square, this syntax \textbf{solves the linear system} $Ax = b$.
Therefore, it returns \texttt{x} such that \texttt{A * x = b}.
If \texttt{A} is rectangular, it \textbf{solves for the least-squares
solution} to the
problem.
\\\hline
\end{longtable} 

\section{Programming}
The following are useful basics for Julia programming.
\begin{longtable}{ |m{6cm}  | m{11cm} |}
	\hline
	\textbf{Command} & \textbf{Description}
	\\\hline
	\begin{minted}{julia}
    # One line comment

    #=
    Comment 
    block
    =#
	\end{minted}
	& Two ways to make \textbf{comments}. Comments are useful for annotating
    code and explaining what it does. The first example limits your comment to
    one line and the second example allows the comments to span multiple lines
    between the \texttt{\#=} and \texttt{=\#}.
    \\\hline
\begin{minted}{julia}
    for i in iterable
        # do something
    end
	\end{minted}
	& A \textbf{for loop} is used to perform a sequence of commands for each
    element in an iterable object, such as an array. For example, the
    following for loop fills the vector \texttt{l} with the squares of the
    integers from 1 to 3:
    \begin{minted}{julia}
    N = 3
    l = zeros(N, 1)
    for i = 1:N
        l[i] = i ^ 2
    end
    \end{minted}
    \\\hline
\begin{minted}{julia}
    while i <= N
        # do something
    end
	\end{minted}
	& A \textbf{while loop} performs a sequence of commands as long as some
    condition is true. For example, the
    following while loop achieves the same result as the for loop above
    \begin{minted}{julia}
    N = 3
    l = zeros(N, 1)
    i = 1
    while i <= N
        l[i] = i ^ 2
        i = i + 1
    end
    \end{minted}
    \\\hline
\begin{minted}{julia}
    if i <= N
        # do something
    else
        # do something else
    end
	\end{minted}
	& An \textbf{if/else statement} performs commands if a condition is met. For example, the
    following squares \texttt{x} is \texttt{x} is 5, and cubes it otherwise:
    \begin{minted}{julia}
    if x == 5
        x = x ^ 2
    else
        x = x ^ 3
    end
    \end{minted}
    We can also just have an if statement on its own, in which case it would
    square \texttt{x} if \texttt{x} is 5, and do nothing otherwise.
    \begin{minted}{julia}
    if x == 5
        x = x ^ 2
    end
    \end{minted}
    \\\hline
\begin{minted}{julia}
    fun(x, y) = 5 * x + y

    function fun(x, y)
        ret = 5 * x
        return ret + y
    end
	\end{minted}
	& These are two ways to define \textbf{functions}. Both examples here define
    equivalent functions.

    The first method is for defining a function on one line. The name of the
    function is \texttt{fun} and it takes two inputs, \texttt{x} and
    \texttt{y}, which are specified between the parentheses. The code after
    the equals sign tells Julia what the output of the function is.

    The second method is used to create functions of more than one line. The
    name of the function, \texttt{fun}, is specified right after
    \texttt{function}, and like the one-line version, has its arguments in
    parentheses. The \texttt{return} statement specifies the output of the
    function.
    \\\hline
\begin{minted}{julia}
    println("Hello world")
	\end{minted}
	& How to \textbf{print} to screen. We can also print the values of
    variables to screen:
    \begin{minted}{julia}
    println("The value of x is $(x).")
    \end{minted}
    \\\hline

\end{longtable}


\end{document}
